\documentclass[dvipdfmx]{standalone}
\usepackage{circuitikz}
\usepackage{tikz}
\usetikzlibrary{calc} % $(A)+(dx,dy)$ の座標計算用

\begin{document}
% ====== 全体の“ノリ”はここで決める(数字はここだけ) ======
\newcommand{\vv}{0.9}   % 縦方向の標準スパン(↑↓)
\newcommand{\hh}{1.0}   % 横方向の標準スパン(←→)
\ctikzset{
  bipoles/length=1.1cm,        % 2端子素子の見た目の長さ
  label distance=2mm,          % ラベルを少し離す
  % bipoles/thickness=2        % 線・素子を太くしたいとき
}

\begin{circuitikz}[american voltages, scale=1.0]
  % 見やすさ設定(任意)
  \ctikzset{
    bipoles/length=1.1cm,
    label distance=2mm,
  }

  % ===== 電源:下端がGND、上端がVCCレール =====
  \draw (0,0) node[ground](GND){}                  % GNDノードを明示
        to[battery1,l=$V_{CC}$] (0,4.2)           % バッテリ上端をVCCへ
        to[short] (3.6,4.2) coordinate (VCCrail); % 上レール(右へ)

  % 出力端子(上レールの右端)
  \draw (VCCrail) node[ocirc,label=right:$v_{out}$]{};

  % ===== コレクタ側:上レール → RC → トランジスタ =====
  % 上レールから「下に間」を作ってRC、その下にトランジスタのコレクタ
  \draw (2.0,4.2) to[short] (2.0,3.6)
                  to[R,l_=$R_C$, i>^=$I_C$] (2.0,2.8);

  % トランジスタ本体:コレクタ位置を(2.0,2.4)
  \draw (2.0,2.4) node[npn,anchor=collector] (Q1) {};

  % コレクタ端子をRC下端に直結(完全に縦)
  \draw (Q1.collector) to[short] (2.0,2.8);

  % ===== エミッタ側:トランジスタ → RE → GND =====
  % エミッタから下へ抵抗→さらに下でGND(左のGNDノードへ)
  \draw (Q1.emitter) to[R,l_=$R_E$] (2.0,0.6)
                     to[short] (2.0,0) to[short] (GND);

  % ===== ベース側:入力源 → RB → ベース =====
  % 入力源は左下、y=1.2の水平ラインに沿ってベースへ
  \draw (0,1.2) to[sinusoidal voltage source,l=$v_{in}$] (0,0);   % 入力源→GND
  \draw (1.0,1.2) to[R,l=$R_B$] (0,1.2);                          % RB
  \draw (1.0,1.2) to[short] (1.8,1.2) to[short] (Q1.base);         % ベース接続

  % 素子ラベル
  \node[right] at (2.4,2.4) {$Q_1$ (NPN)};
\end{circuitikz}



\end{document}