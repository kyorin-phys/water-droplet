\documentclass[tikz,border=5pt]{standalone}
\usepackage{circuitikz} % 追加パッケージなし

%--------- FET 切替(環境により nmos か nfet かが異なる)---------
\def\FET{nfet}   % ← もし未定義エラーが出たら「nmos」に変えて再コンパイル
%\def\FET{nmos}
%-------------------------------------------------------------------

\begin{document}
\begin{circuitikz}

%--------------------------------
% 1) チップ本体(dipchip, 30ピン)
%--------------------------------
\draw (0,0) node[dipchip, num pins=30] (U) {};
% チップ名は枠の上(above)
\node[above=2mm] at (U.north) {AE-RP2040};

% 使用ピンの注記(視覚補助)
\node[right=2mm] at (U.pin 20) {Pin 20 / GP16};
\node[right=2mm] at (U.pin 21) {Pin 21 / GP15};
% 「左側の Pin 23 / GND」を視覚整合として左ピンに注記(pin 10 を流用)
\node[left=2mm]  at (U.pin 10) {Pin 23 / GND};

%--------------------------------
% 2) GNDバス(左→大きく下→右:チップやラベルと非交差)
%--------------------------------
\draw (U.pin 10) -- ++(-0.8,0) coordinate(gL)
                 -- ++(0,-5.0) coordinate(gD)
                 -- ++(13.0,0) coordinate(gR); % 右向き水平 GND バス

%--------------------------------
% 3) 上段:GP16 → ソレノイド駆動(1kΩ → N-FET → ソレノイド || ダイオード → +12V)
%--------------------------------
% ピン20をいったん上へ持ち上げて右へ(交差回避)
\draw (U.pin 20) -- ++(0,0.8) coordinate(p20up) -- ++(2.2,0)
      to[R=$1\,\mathrm{k}\Omega$] ++(1.8,0) coordinate(gatein);

% MOSFET(環境差に対応:\FET で nmos / nfet を切替)
\draw (gatein) node[\FET, anchor=G] (M1) {};

% ソース → 縦落とし → 右の GND バスへ
\draw (M1.S) -- ++(0,-0.8) coordinate(m1s_drop)
             -- (m1s_drop |- gR) -- (gR);

% ドレイン上:ソレノイド → 上で +12V ノード
\draw (M1.D) -- ++(0,0.6) coordinate(sol_low)
      to[L, l_=Solenoid] ++(0,2.2) coordinate(sol_top);
\draw (sol_top) to[short,-o] ++(0,0.6) node[above] {$+12\,\mathrm{V}$};

% ソレノイドと並列の保護ダイオード(アノード下、カソード上)
\draw (sol_low) to[diode, l=$1\mathrm{N}4007$,*-*] (sol_top);

%--------------------------------
% 4) 下段:GP15 → カメラ制御(PC817 相当:入力 LED + 出力 NPN)
%--------------------------------
% ピン21をいったん下へ落として右へ(交差回避)
\draw (U.pin 21) -- ++(0,-0.8) coordinate(p21down) -- ++(2.2,0)
      to[R=$330\,\Omega$] ++(1.8,0) coordinate(led_in);

% 入力側:LED相当(互換性重視のため diode 記号に"LED"ラベル)
\draw (led_in) to[diode, l=LED] ++(0,-1.6) coordinate(led_gnd_drop);
% LED カソード → 縦落とし → 右の GND バスへ
\draw (led_gnd_drop) -- (led_gnd_drop |- gR) -- (gR);

% 出力側:PC817 のトランジスタ相当(NPN)
\draw (led_in) ++(2.6,0.0) node[npn, anchor=B] (Qopto) {};

% エミッタ → Camera GND(外部端子)→ 内部的に GND バスへ
\draw (Qopto.E) -- ++(0,-0.8) coordinate(cam_gnd_tap)
      to[short,-o] ++(1.4,0) node[right]{Camera GND};
\draw (cam_gnd_tap) -- (cam_gnd_tap |- gR) -- (gR);

% コレクタ → Camera Tip(外部端子)
\draw (Qopto.C) -- ++(1.2,0) to[short,-o] ++(1.0,0) node[right]{Camera Tip};

% PC817 の枠(点線ボックス)とラベル
\draw[dashed, rounded corners=2pt]
      ($(led_in)+(0.3,0.5)$) rectangle ($(Qopto.C)+(1.4,-1.3)$);
\node at ([xshift=4mm,yshift=-2mm]led_in) {PC817};

%--------------------------------
% 5) 参照ラベル(新構文を使わず x/yshift で配置)
%--------------------------------
\node at ([xshift=-2mm,yshift=0mm]p20up) {\footnotesize GP16};
\node at ([xshift=-2mm,yshift=0mm]p21down) {\footnotesize GP15};

\end{circuitikz}
\end{document}